% Options for packages loaded elsewhere
\PassOptionsToPackage{unicode}{hyperref}
\PassOptionsToPackage{hyphens}{url}
\PassOptionsToPackage{dvipsnames,svgnames,x11names}{xcolor}
%
\documentclass[
  letterpaper,
  DIV=11,
  numbers=noendperiod]{scrartcl}

\usepackage{amsmath,amssymb}
\usepackage{iftex}
\ifPDFTeX
  \usepackage[T1]{fontenc}
  \usepackage[utf8]{inputenc}
  \usepackage{textcomp} % provide euro and other symbols
\else % if luatex or xetex
  \usepackage{unicode-math}
  \defaultfontfeatures{Scale=MatchLowercase}
  \defaultfontfeatures[\rmfamily]{Ligatures=TeX,Scale=1}
\fi
\usepackage{lmodern}
\ifPDFTeX\else  
    % xetex/luatex font selection
\fi
% Use upquote if available, for straight quotes in verbatim environments
\IfFileExists{upquote.sty}{\usepackage{upquote}}{}
\IfFileExists{microtype.sty}{% use microtype if available
  \usepackage[]{microtype}
  \UseMicrotypeSet[protrusion]{basicmath} % disable protrusion for tt fonts
}{}
\makeatletter
\@ifundefined{KOMAClassName}{% if non-KOMA class
  \IfFileExists{parskip.sty}{%
    \usepackage{parskip}
  }{% else
    \setlength{\parindent}{0pt}
    \setlength{\parskip}{6pt plus 2pt minus 1pt}}
}{% if KOMA class
  \KOMAoptions{parskip=half}}
\makeatother
\usepackage{xcolor}
\setlength{\emergencystretch}{3em} % prevent overfull lines
\setcounter{secnumdepth}{-\maxdimen} % remove section numbering
% Make \paragraph and \subparagraph free-standing
\ifx\paragraph\undefined\else
  \let\oldparagraph\paragraph
  \renewcommand{\paragraph}[1]{\oldparagraph{#1}\mbox{}}
\fi
\ifx\subparagraph\undefined\else
  \let\oldsubparagraph\subparagraph
  \renewcommand{\subparagraph}[1]{\oldsubparagraph{#1}\mbox{}}
\fi

\usepackage{color}
\usepackage{fancyvrb}
\newcommand{\VerbBar}{|}
\newcommand{\VERB}{\Verb[commandchars=\\\{\}]}
\DefineVerbatimEnvironment{Highlighting}{Verbatim}{commandchars=\\\{\}}
% Add ',fontsize=\small' for more characters per line
\usepackage{framed}
\definecolor{shadecolor}{RGB}{241,243,245}
\newenvironment{Shaded}{\begin{snugshade}}{\end{snugshade}}
\newcommand{\AlertTok}[1]{\textcolor[rgb]{0.68,0.00,0.00}{#1}}
\newcommand{\AnnotationTok}[1]{\textcolor[rgb]{0.37,0.37,0.37}{#1}}
\newcommand{\AttributeTok}[1]{\textcolor[rgb]{0.40,0.45,0.13}{#1}}
\newcommand{\BaseNTok}[1]{\textcolor[rgb]{0.68,0.00,0.00}{#1}}
\newcommand{\BuiltInTok}[1]{\textcolor[rgb]{0.00,0.23,0.31}{#1}}
\newcommand{\CharTok}[1]{\textcolor[rgb]{0.13,0.47,0.30}{#1}}
\newcommand{\CommentTok}[1]{\textcolor[rgb]{0.37,0.37,0.37}{#1}}
\newcommand{\CommentVarTok}[1]{\textcolor[rgb]{0.37,0.37,0.37}{\textit{#1}}}
\newcommand{\ConstantTok}[1]{\textcolor[rgb]{0.56,0.35,0.01}{#1}}
\newcommand{\ControlFlowTok}[1]{\textcolor[rgb]{0.00,0.23,0.31}{#1}}
\newcommand{\DataTypeTok}[1]{\textcolor[rgb]{0.68,0.00,0.00}{#1}}
\newcommand{\DecValTok}[1]{\textcolor[rgb]{0.68,0.00,0.00}{#1}}
\newcommand{\DocumentationTok}[1]{\textcolor[rgb]{0.37,0.37,0.37}{\textit{#1}}}
\newcommand{\ErrorTok}[1]{\textcolor[rgb]{0.68,0.00,0.00}{#1}}
\newcommand{\ExtensionTok}[1]{\textcolor[rgb]{0.00,0.23,0.31}{#1}}
\newcommand{\FloatTok}[1]{\textcolor[rgb]{0.68,0.00,0.00}{#1}}
\newcommand{\FunctionTok}[1]{\textcolor[rgb]{0.28,0.35,0.67}{#1}}
\newcommand{\ImportTok}[1]{\textcolor[rgb]{0.00,0.46,0.62}{#1}}
\newcommand{\InformationTok}[1]{\textcolor[rgb]{0.37,0.37,0.37}{#1}}
\newcommand{\KeywordTok}[1]{\textcolor[rgb]{0.00,0.23,0.31}{#1}}
\newcommand{\NormalTok}[1]{\textcolor[rgb]{0.00,0.23,0.31}{#1}}
\newcommand{\OperatorTok}[1]{\textcolor[rgb]{0.37,0.37,0.37}{#1}}
\newcommand{\OtherTok}[1]{\textcolor[rgb]{0.00,0.23,0.31}{#1}}
\newcommand{\PreprocessorTok}[1]{\textcolor[rgb]{0.68,0.00,0.00}{#1}}
\newcommand{\RegionMarkerTok}[1]{\textcolor[rgb]{0.00,0.23,0.31}{#1}}
\newcommand{\SpecialCharTok}[1]{\textcolor[rgb]{0.37,0.37,0.37}{#1}}
\newcommand{\SpecialStringTok}[1]{\textcolor[rgb]{0.13,0.47,0.30}{#1}}
\newcommand{\StringTok}[1]{\textcolor[rgb]{0.13,0.47,0.30}{#1}}
\newcommand{\VariableTok}[1]{\textcolor[rgb]{0.07,0.07,0.07}{#1}}
\newcommand{\VerbatimStringTok}[1]{\textcolor[rgb]{0.13,0.47,0.30}{#1}}
\newcommand{\WarningTok}[1]{\textcolor[rgb]{0.37,0.37,0.37}{\textit{#1}}}

\providecommand{\tightlist}{%
  \setlength{\itemsep}{0pt}\setlength{\parskip}{0pt}}\usepackage{longtable,booktabs,array}
\usepackage{calc} % for calculating minipage widths
% Correct order of tables after \paragraph or \subparagraph
\usepackage{etoolbox}
\makeatletter
\patchcmd\longtable{\par}{\if@noskipsec\mbox{}\fi\par}{}{}
\makeatother
% Allow footnotes in longtable head/foot
\IfFileExists{footnotehyper.sty}{\usepackage{footnotehyper}}{\usepackage{footnote}}
\makesavenoteenv{longtable}
\usepackage{graphicx}
\makeatletter
\def\maxwidth{\ifdim\Gin@nat@width>\linewidth\linewidth\else\Gin@nat@width\fi}
\def\maxheight{\ifdim\Gin@nat@height>\textheight\textheight\else\Gin@nat@height\fi}
\makeatother
% Scale images if necessary, so that they will not overflow the page
% margins by default, and it is still possible to overwrite the defaults
% using explicit options in \includegraphics[width, height, ...]{}
\setkeys{Gin}{width=\maxwidth,height=\maxheight,keepaspectratio}
% Set default figure placement to htbp
\makeatletter
\def\fps@figure{htbp}
\makeatother

\usepackage{booktabs}
\usepackage{longtable}
\usepackage{array}
\usepackage{multirow}
\usepackage{wrapfig}
\usepackage{float}
\usepackage{colortbl}
\usepackage{pdflscape}
\usepackage{tabu}
\usepackage{threeparttable}
\usepackage{threeparttablex}
\usepackage[normalem]{ulem}
\usepackage{makecell}
\usepackage{xcolor}
\KOMAoption{captions}{tableheading}
\makeatletter
\makeatother
\makeatletter
\makeatother
\makeatletter
\@ifpackageloaded{caption}{}{\usepackage{caption}}
\AtBeginDocument{%
\ifdefined\contentsname
  \renewcommand*\contentsname{Table of contents}
\else
  \newcommand\contentsname{Table of contents}
\fi
\ifdefined\listfigurename
  \renewcommand*\listfigurename{List of Figures}
\else
  \newcommand\listfigurename{List of Figures}
\fi
\ifdefined\listtablename
  \renewcommand*\listtablename{List of Tables}
\else
  \newcommand\listtablename{List of Tables}
\fi
\ifdefined\figurename
  \renewcommand*\figurename{Figure}
\else
  \newcommand\figurename{Figure}
\fi
\ifdefined\tablename
  \renewcommand*\tablename{Table}
\else
  \newcommand\tablename{Table}
\fi
}
\@ifpackageloaded{float}{}{\usepackage{float}}
\floatstyle{ruled}
\@ifundefined{c@chapter}{\newfloat{codelisting}{h}{lop}}{\newfloat{codelisting}{h}{lop}[chapter]}
\floatname{codelisting}{Listing}
\newcommand*\listoflistings{\listof{codelisting}{List of Listings}}
\makeatother
\makeatletter
\@ifpackageloaded{caption}{}{\usepackage{caption}}
\@ifpackageloaded{subcaption}{}{\usepackage{subcaption}}
\makeatother
\makeatletter
\@ifpackageloaded{tcolorbox}{}{\usepackage[skins,breakable]{tcolorbox}}
\makeatother
\makeatletter
\@ifundefined{shadecolor}{\definecolor{shadecolor}{rgb}{.97, .97, .97}}
\makeatother
\makeatletter
\makeatother
\makeatletter
\makeatother
\ifLuaTeX
  \usepackage{selnolig}  % disable illegal ligatures
\fi
\IfFileExists{bookmark.sty}{\usepackage{bookmark}}{\usepackage{hyperref}}
\IfFileExists{xurl.sty}{\usepackage{xurl}}{} % add URL line breaks if available
\urlstyle{same} % disable monospaced font for URLs
\hypersetup{
  pdftitle={Econometría Financiera},
  pdfauthor={José Antonio Burgos Francisco},
  colorlinks=true,
  linkcolor={blue},
  filecolor={Maroon},
  citecolor={Blue},
  urlcolor={Blue},
  pdfcreator={LaTeX via pandoc}}

\title{Econometría Financiera}
\usepackage{etoolbox}
\makeatletter
\providecommand{\subtitle}[1]{% add subtitle to \maketitle
  \apptocmd{\@title}{\par {\large #1 \par}}{}{}
}
\makeatother
\subtitle{Práctica 1: Estimaciones de modelos ARIMA}
\author{José Antonio Burgos Francisco}
\date{}

\begin{document}
\maketitle
\ifdefined\Shaded\renewenvironment{Shaded}{\begin{tcolorbox}[boxrule=0pt, interior hidden, borderline west={3pt}{0pt}{shadecolor}, sharp corners, enhanced, frame hidden, breakable]}{\end{tcolorbox}}\fi

\renewcommand*\contentsname{Índice de contenidos}
{
\hypersetup{linkcolor=}
\setcounter{tocdepth}{3}
\tableofcontents
}
\newpage

\hypertarget{visualizaciuxf3n-de-las-series}{%
\section{Visualización de las
Series}\label{visualizaciuxf3n-de-las-series}}

\hypertarget{gruxe1fico1-series-en-logaritmos}{%
\subsection{Gráfico1: Series en
logaritmos}\label{gruxe1fico1-series-en-logaritmos}}

\includegraphics{trabajo1-jose-burgos_files/figure-pdf/plot_ipc_log-1.pdf}

\hypertarget{gruxe1fico-2-series-en-primeras-diferencias}{%
\subsection{Gráfico 2: Series en primeras
diferencias}\label{gruxe1fico-2-series-en-primeras-diferencias}}

\includegraphics{trabajo1-jose-burgos_files/figure-pdf/plot_ipc_diff1-1.pdf}

\hypertarget{gruxe1fico-3-series-en-variaciuxf3n-interanual}{%
\subsection{Gráfico 3: Series en Variación
Interanual}\label{gruxe1fico-3-series-en-variaciuxf3n-interanual}}

\includegraphics{trabajo1-jose-burgos_files/figure-pdf/plot_ipc_vi-1.pdf}

\newpage

\hypertarget{anuxe1lisis-de-rauxedz-unitaria}{%
\section{Análisis de Raíz
Unitaria}\label{anuxe1lisis-de-rauxedz-unitaria}}

\hypertarget{prueba-dickey-fuller-aumentada-adf.}{%
\subsection{Prueba Dickey-Fuller Aumentada
(ADF).}\label{prueba-dickey-fuller-aumentada-adf.}}

\textbf{Tabla 1: Resultados del test de Dickey-Fuller aumentado (ADF):
Series en logaritmo}

\begin{longtable}[]{@{}
  >{\raggedright\arraybackslash}p{(\columnwidth - 8\tabcolsep) * \real{0.2000}}
  >{\raggedleft\arraybackslash}p{(\columnwidth - 8\tabcolsep) * \real{0.2000}}
  >{\raggedleft\arraybackslash}p{(\columnwidth - 8\tabcolsep) * \real{0.2000}}
  >{\raggedleft\arraybackslash}p{(\columnwidth - 8\tabcolsep) * \real{0.2000}}
  >{\raggedright\arraybackslash}p{(\columnwidth - 8\tabcolsep) * \real{0.2000}}@{}}
\toprule\noalign{}
\begin{minipage}[b]{\linewidth}\raggedright
Serie
\end{minipage} & \begin{minipage}[b]{\linewidth}\raggedleft
Estadístico ADF (\(\tau_3\))
\end{minipage} & \begin{minipage}[b]{\linewidth}\raggedleft
Valor crítico 5 \%
\end{minipage} & \begin{minipage}[b]{\linewidth}\raggedleft
Nivel de significancia
\end{minipage} & \begin{minipage}[b]{\linewidth}\raggedright
Conclusión
\end{minipage} \\
\midrule\noalign{}
\endhead
\bottomrule\noalign{}
\endlastfoot
IPC & -2.013 & -3.43 & 5 \% & No estacionaria \\
IPC subyacente & -1.813 & -3.43 & 5 \% & No estacionaria \\
IPC A\&B & -2.088 & -3.43 & 5 \% & No estacionaria \\
IPC transporte & -2.624 & -3.43 & 5 \% & No estacionaria \\
\end{longtable}

\emph{Los resultados del test de Dickey-Fuller aumentado (ADF) indican
que ninguna de las series en niveles rechaza la hipótesis nula de
presencia de raíz unitaria al 5 \% de significancia, ver
\protect\hyperlink{apendice-a}{Apéndice A}. En consecuencia, se concluye
que las series no son estacionarias en niveles. Se procede a transformar
las series mediante diferenciación de primer orden, con el objetivo de
inducir estacionariedad.}

\hypertarget{transformaciuxf3n-de-serie-en-primera-diferencia}{%
\subsection{Transformación de serie en primera
diferencia}\label{transformaciuxf3n-de-serie-en-primera-diferencia}}

\begin{Shaded}
\begin{Highlighting}[]
\NormalTok{ipcs\_log\_diff }\OtherTok{\textless{}{-}} \FunctionTok{map}\NormalTok{(ipc\_log, diff)     }\CommentTok{\# Primera diferencia de logaritmos}
\NormalTok{test\_diff }\OtherTok{\textless{}{-}} \FunctionTok{map}\NormalTok{(ipcs\_log\_diff, }\SpecialCharTok{\textasciitilde{}}\NormalTok{urca}\SpecialCharTok{::}\FunctionTok{ur.df}\NormalTok{(.x, }\AttributeTok{type =} \StringTok{"trend"}\NormalTok{, }\AttributeTok{lags =} \DecValTok{4}\NormalTok{))}
\end{Highlighting}
\end{Shaded}

\textbf{Tabla 2: Resultados del test de Dickey-Fuller aumentado (ADF):
Series con primeras diferencias}

\begin{longtable}[]{@{}
  >{\raggedright\arraybackslash}p{(\columnwidth - 8\tabcolsep) * \real{0.2188}}
  >{\raggedleft\arraybackslash}p{(\columnwidth - 8\tabcolsep) * \real{0.2083}}
  >{\raggedleft\arraybackslash}p{(\columnwidth - 8\tabcolsep) * \real{0.1979}}
  >{\raggedleft\arraybackslash}p{(\columnwidth - 8\tabcolsep) * \real{0.2500}}
  >{\raggedright\arraybackslash}p{(\columnwidth - 8\tabcolsep) * \real{0.1250}}@{}}
\toprule\noalign{}
\begin{minipage}[b]{\linewidth}\raggedright
Serie
\end{minipage} & \begin{minipage}[b]{\linewidth}\raggedleft
Estadístico ADF (\(\tau_3\))
\end{minipage} & \begin{minipage}[b]{\linewidth}\raggedleft
Valor crítico 5 \%
\end{minipage} & \begin{minipage}[b]{\linewidth}\raggedleft
Nivel de significancia
\end{minipage} & \begin{minipage}[b]{\linewidth}\raggedright
Conclusión
\end{minipage} \\
\midrule\noalign{}
\endhead
\bottomrule\noalign{}
\endlastfoot
IPC & -6.043 & -3.43 & 5 \% & Estacionaria \\
IPC subyacente & -3.902 & -3.43 & 5 \% & Estacionaria \\
IPC A\&B & -6.406 & -3.43 & 5 \% & Estacionaria \\
IPC transporte & -7.654 & -3.43 & 5 \% & Estacionaria \\
\end{longtable}

\emph{Luego de aplicar la transformación en primeras diferencias, los
resultados del test ADF indican que todas las series rechazan la
hipótesis nula de raíz unitaria al 5 \% de significancia. En
consecuencia, se concluye que las series transformadas son
estacionarias. Para más detalles del test véase el Apéndice B.}

\hypertarget{preparaciuxf3n-de-los-datos}{%
\section{Preparación de los datos}\label{preparaciuxf3n-de-los-datos}}

\hypertarget{estaduxedsticas-descriptivas}{%
\subsection{Estadísticas
descriptivas}\label{estaduxedsticas-descriptivas}}

\textbf{Tabla 3: Estadísticos descriptivos de las primeras diferencias
logarítmicas del IPC}

\begin{longtable}[]{@{}lrrrr@{}}
\toprule\noalign{}
Serie & Media & Varianza & Mínimo & Máximo \\
\midrule\noalign{}
\endhead
\bottomrule\noalign{}
\endlastfoot
IPC & 0.0038 & 0.0000308 & -0.0333 & 0.0235 \\
IPC subyacente & 0.0034 & 0.00000575 & -0.0011 & 0.0138 \\
IPC A\&B & 0.0048 & 0.0000810 & -0.0211 & 0.0380 \\
IPC transporte & 0.0038 & 0.0003130 & -0.1250 & 0.0641 \\
\end{longtable}

\hypertarget{correlaciones}{%
\subsection{Correlaciones}\label{correlaciones}}

\textbf{Tabla 4: Matriz de correlaciones entre las primeras diferencias
logarítmicas del IPC}

\begin{longtable}[]{@{}lrrrr@{}}
\toprule\noalign{}
& IPC & IPC subyacente & IPC A\&B & IPC transporte \\
\midrule\noalign{}
\endhead
\bottomrule\noalign{}
\endlastfoot
IPC & 1.000 & & & \\
IPC subyacente & 0.39 & 1.000 & & \\
IPC A\&B & 0.60 & 0.40 & 1.000 & \\
IPC transporte & 0.84 & 0.06 & 0.14 & 1.000 \\
\end{longtable}

\hypertarget{gruxe1fico-4-primeras-diferencias-logaruxedtmicas-del-ipc}{%
\subsection{Gráfico 4: Primeras diferencias logarítmicas del
IPC}\label{gruxe1fico-4-primeras-diferencias-logaruxedtmicas-del-ipc}}

\includegraphics{trabajo1-jose-burgos_files/figure-pdf/unnamed-chunk-5-1.pdf}

\hypertarget{identificaciuxf3n-del-modelo-arima}{%
\section{Identificación del Modelo
ARIMA}\label{identificaciuxf3n-del-modelo-arima}}

\hypertarget{gruxe1fico-5-funciuxf3n-de-autocorrelaciuxf3n-simple-acf}{%
\subsection{Gráfico 5: Función de autocorrelación simple
(ACF)}\label{gruxe1fico-5-funciuxf3n-de-autocorrelaciuxf3n-simple-acf}}

\includegraphics{trabajo1-jose-burgos_files/figure-pdf/unnamed-chunk-6-1.pdf}

\hypertarget{gruxe1fico-6-funciuxf3n-de-autocorrelaciuxf3n-parcial-pacf}{%
\subsection{Gráfico 6: Función de autocorrelación parcial
(PACF)}\label{gruxe1fico-6-funciuxf3n-de-autocorrelaciuxf3n-parcial-pacf}}

\includegraphics{trabajo1-jose-burgos_files/figure-pdf/unnamed-chunk-7-1.pdf}

\hypertarget{justificaciuxf3n-de-modelo-arma-apropiado}{%
\subsubsection{Justificación de modelo ARMA
apropiado}\label{justificaciuxf3n-de-modelo-arma-apropiado}}

\emph{El análisis de las funciones de autocorrelación (ACF) y
autocorrelación parcial (PACF) de las primeras diferencias logarítmicas
sugiere que las series presentan una dinámica de corto plazo dominada
por la persistencia en el primer rezago. En particular, para el IPC
general, IPC de alimentos y bebidas, y el IPC de transporte, la PACF
muestra un corte claro en el primer rezago, mientras que la ACF decae
gradualmente, lo cual es consistente con un proceso AR(1).}

\hypertarget{estimaciuxf3n-del-modelo}{%
\section{Estimación del modelo}\label{estimaciuxf3n-del-modelo}}

\hypertarget{resultados-de-la-estimaciuxf3n}{%
\subsection{Resultados de la
estimación}\label{resultados-de-la-estimaciuxf3n}}

\begin{Shaded}
\begin{Highlighting}[]
\NormalTok{modelos\_arma }\OtherTok{\textless{}{-}}\NormalTok{ ipc\_to\_model }\SpecialCharTok{|\textgreater{}} 
  \FunctionTok{group\_by}\NormalTok{(key) }\SpecialCharTok{|\textgreater{}} 
  \FunctionTok{model}\NormalTok{(}\AttributeTok{arma11 =} \FunctionTok{ARIMA}\NormalTok{(value }\SpecialCharTok{\textasciitilde{}} \FunctionTok{pdq}\NormalTok{(}\DecValTok{1}\NormalTok{, }\DecValTok{0}\NormalTok{, }\DecValTok{1}\NormalTok{)))}
\end{Highlighting}
\end{Shaded}

\textbf{Tabla 5: Resultados de la estimación de modelos ARMA(1,1) para
las primeras diferencias logarítmicas del IPC}

\begin{longtable}[]{@{}
  >{\raggedright\arraybackslash}p{(\columnwidth - 10\tabcolsep) * \real{0.2394}}
  >{\centering\arraybackslash}p{(\columnwidth - 10\tabcolsep) * \real{0.1127}}
  >{\raggedleft\arraybackslash}p{(\columnwidth - 10\tabcolsep) * \real{0.2676}}
  >{\raggedleft\arraybackslash}p{(\columnwidth - 10\tabcolsep) * \real{0.1268}}
  >{\raggedleft\arraybackslash}p{(\columnwidth - 10\tabcolsep) * \real{0.1268}}
  >{\raggedleft\arraybackslash}p{(\columnwidth - 10\tabcolsep) * \real{0.1268}}@{}}
\toprule\noalign{}
\begin{minipage}[b]{\linewidth}\raggedright
Serie
\end{minipage} & \begin{minipage}[b]{\linewidth}\centering
Modelo
\end{minipage} & \begin{minipage}[b]{\linewidth}\raggedleft
Log-verosimilitud
\end{minipage} & \begin{minipage}[b]{\linewidth}\raggedleft
AIC
\end{minipage} & \begin{minipage}[b]{\linewidth}\raggedleft
AICc
\end{minipage} & \begin{minipage}[b]{\linewidth}\raggedleft
BIC
\end{minipage} \\
\midrule\noalign{}
\endhead
\bottomrule\noalign{}
\endlastfoot
IPC & ARMA(1,1) & 925.00 & -1842.00 & -1842.00 & -1828.00 \\
IPC subyacente & ARMA(1,1) & 1160.00 & -2308.00 & -2308.00 & -2287.00 \\
IPC A\&B & ARMA(1,1) & 807.00 & -1604.00 & -1604.00 & -1587.00 \\
IPC transporte & ARMA(1,1) & 654.00 & -1295.00 & -1295.00 & -1274.00 \\
\end{longtable}

\hypertarget{evaluaciuxf3n-de-los-modelos}{%
\subsection{Evaluación de los
modelos}\label{evaluaciuxf3n-de-los-modelos}}

\hypertarget{gruxe1fico-6-comportamiento-de-los-residuos}{%
\subsubsection{Gráfico 6: Comportamiento de los
residuos}\label{gruxe1fico-6-comportamiento-de-los-residuos}}

\includegraphics{trabajo1-jose-burgos_files/figure-pdf/unnamed-chunk-9-1.pdf}

\hypertarget{gruxe1fico-8-acf-de-los-residuos}{%
\subsubsection{Gráfico 8: ACF de los
residuos}\label{gruxe1fico-8-acf-de-los-residuos}}

\includegraphics{trabajo1-jose-burgos_files/figure-pdf/unnamed-chunk-10-1.pdf}

\emph{La función de autocorrelación (ACF) de los residuos de los modelos
ARMA(1,1) no muestra patrones sistemáticos ni picos persistentes a lo
largo de los rezagos considerados. En la mayoría de los casos, las
autocorrelaciones se mantienen dentro de las bandas de confianza, lo que
indica ausencia de dependencia serial remanente en los residuos.}

\hypertarget{gruxe1fico-9-histograma-de-los-residuos}{%
\subsubsection{Gráfico 9: Histograma de los
residuos}\label{gruxe1fico-9-histograma-de-los-residuos}}

\includegraphics{trabajo1-jose-burgos_files/figure-pdf/unnamed-chunk-11-1.pdf}

\emph{La distribución de los residuos de los modelos ARMA(1,1) muestra
una forma aproximadamente simétrica y centrada en torno a cero para
todas las series, lo que sugiere que el modelo captura adecuadamente la
dinámica media de los datos.}

\hypertarget{pruebas-estaduxedsticas}{%
\subsubsection{Pruebas estadísticas}\label{pruebas-estaduxedsticas}}

\textbf{Tabla 6: Test Ljung-Box autocorrelación en residuos}

\begin{longtable*}[t]{lccc}
\toprule
Serie & Modelo & Estadístico Q & p-valor\\
\midrule
ipc & arma11 & 11.323 & 0.501\\
ipc\_ayb & arma11 & 15.670 & 0.207\\
ipc\_subyacente & arma11 & 11.577 & 0.480\\
ipc\_transporte & arma11 & 18.283 & 0.107\\
\bottomrule
\end{longtable*}

\emph{La prueba de Ljung--Box aplicada a los residuos de los modelos
ARMA(1,1) no rechaza la hipótesis nula de ausencia de autocorrelación
hasta el rezago 12 en ninguna de las series analizadas (p-valores
mayores a 0.10). Este resultado sugiere que los residuos se comportan
como ruido blanco, lo que respalda la adecuada especificación de los
modelos estimados.}

\hypertarget{conclusiuxf3n}{%
\section{Conclusión}\label{conclusiuxf3n}}

\emph{Los resultados muestran que las variaciones del IPC y de sus
componentes pueden describirse adecuadamente mediante modelos ARMA(1,1),
lo que evidencia la existencia de dependencia temporal de corto plazo en
las series. El IPC subyacente presenta mayor persistencia, mientras que
los componentes de A\&B y transporte responden con mayor intensidad a
choques transitorios. Los criterios de información confirman la
idoneidad de los modelos estimados y el análisis de residuos no revela
autocorrelación remanente significativa, respaldando la validez de la
especificación adoptada.}

\newpage

\hypertarget{apuxe9ndices}{%
\section{Apéndices}\label{apuxe9ndices}}

\hypertarget{apuxe9ndice-a-dickey-fuller-aumentada---series-logaritmos}{%
\subsection{Apéndice A: Dickey-Fuller Aumentada - series
logaritmos}\label{apuxe9ndice-a-dickey-fuller-aumentada---series-logaritmos}}

\begin{Shaded}
\begin{Highlighting}[]
\FunctionTok{map}\NormalTok{(test\_log, summary)}
\end{Highlighting}
\end{Shaded}

\begin{verbatim}
$ipc

############################################### 
# Augmented Dickey-Fuller Test Unit Root Test # 
############################################### 

Test regression trend 


Call:
lm(formula = z.diff ~ z.lag.1 + 1 + tt + z.diff.lag)

Residuals:
       Min         1Q     Median         3Q        Max 
-0.0301829 -0.0023644 -0.0000173  0.0021730  0.0191807 

Coefficients:
              Estimate Std. Error t value Pr(>|t|)    
(Intercept)  7.777e-02  3.732e-02   2.084   0.0383 *  
z.lag.1     -1.847e-02  9.172e-03  -2.013   0.0452 *  
tt           6.116e-05  3.184e-05   1.921   0.0560 .  
z.diff.lag1  4.111e-01  6.557e-02   6.270 1.79e-09 ***
z.diff.lag2  1.934e-02  7.095e-02   0.273   0.7855    
z.diff.lag3  2.312e-02  7.095e-02   0.326   0.7449    
z.diff.lag4 -5.155e-02  6.579e-02  -0.784   0.4341    
---
Signif. codes:  0 '***' 0.001 '**' 0.01 '*' 0.05 '.' 0.1 ' ' 1

Residual standard error: 0.005088 on 228 degrees of freedom
Multiple R-squared:  0.1916,    Adjusted R-squared:  0.1703 
F-statistic: 9.007 on 6 and 228 DF,  p-value: 7.749e-09


Value of test-statistic is: -2.0134 9.7486 2.1247 

Critical values for test statistics: 
      1pct  5pct 10pct
tau3 -3.99 -3.43 -3.13
phi2  6.22  4.75  4.07
phi3  8.43  6.49  5.47


$ipc_subyacente

############################################### 
# Augmented Dickey-Fuller Test Unit Root Test # 
############################################### 

Test regression trend 


Call:
lm(formula = z.diff ~ z.lag.1 + 1 + tt + z.diff.lag)

Residuals:
       Min         1Q     Median         3Q        Max 
-0.0072803 -0.0011502 -0.0003454  0.0008352  0.0104388 

Coefficients:
              Estimate Std. Error t value Pr(>|t|)    
(Intercept)  3.837e-02  2.045e-02   1.877 0.061811 .  
z.lag.1     -9.003e-03  4.965e-03  -1.813 0.071096 .  
tt           2.726e-05  1.542e-05   1.767 0.078486 .  
z.diff.lag1  4.055e-01  6.459e-02   6.277 1.72e-09 ***
z.diff.lag2 -2.251e-02  7.011e-02  -0.321 0.748456    
z.diff.lag3  4.616e-02  6.996e-02   0.660 0.510024    
z.diff.lag4  2.212e-01  6.448e-02   3.431 0.000714 ***
---
Signif. codes:  0 '***' 0.001 '**' 0.01 '*' 0.05 '.' 0.1 ' ' 1

Residual standard error: 0.002086 on 228 degrees of freedom
Multiple R-squared:  0.2673,    Adjusted R-squared:  0.248 
F-statistic: 13.86 on 6 and 228 DF,  p-value: 2e-13


Value of test-statistic is: -1.8133 6.7248 1.6713 

Critical values for test statistics: 
      1pct  5pct 10pct
tau3 -3.99 -3.43 -3.13
phi2  6.22  4.75  4.07
phi3  8.43  6.49  5.47


$ipc_ayb

############################################### 
# Augmented Dickey-Fuller Test Unit Root Test # 
############################################### 

Test regression trend 


Call:
lm(formula = z.diff ~ z.lag.1 + 1 + tt + z.diff.lag)

Residuals:
       Min         1Q     Median         3Q        Max 
-0.0216470 -0.0052201  0.0000209  0.0050964  0.0294287 

Coefficients:
              Estimate Std. Error t value Pr(>|t|)    
(Intercept)  1.121e-01  5.200e-02   2.155   0.0322 *  
z.lag.1     -2.785e-02  1.334e-02  -2.088   0.0379 *  
tt           1.248e-04  6.005e-05   2.078   0.0389 *  
z.diff.lag1  4.054e-01  6.539e-02   6.199 2.63e-09 ***
z.diff.lag2 -1.710e-01  6.981e-02  -2.449   0.0151 *  
z.diff.lag3  1.764e-01  6.954e-02   2.537   0.0119 *  
z.diff.lag4 -8.117e-02  6.590e-02  -1.232   0.2194    
---
Signif. codes:  0 '***' 0.001 '**' 0.01 '*' 0.05 '.' 0.1 ' ' 1

Residual standard error: 0.008334 on 228 degrees of freedom
Multiple R-squared:  0.1612,    Adjusted R-squared:  0.1392 
F-statistic: 7.304 on 6 and 228 DF,  p-value: 3.747e-07


Value of test-statistic is: -2.0881 9.6944 2.1821 

Critical values for test statistics: 
      1pct  5pct 10pct
tau3 -3.99 -3.43 -3.13
phi2  6.22  4.75  4.07
phi3  8.43  6.49  5.47


$ipc_transporte

############################################### 
# Augmented Dickey-Fuller Test Unit Root Test # 
############################################### 

Test regression trend 


Call:
lm(formula = z.diff ~ z.lag.1 + 1 + tt + z.diff.lag)

Residuals:
      Min        1Q    Median        3Q       Max 
-0.094862 -0.004228  0.000481  0.006097  0.046152 

Coefficients:
              Estimate Std. Error t value Pr(>|t|)    
(Intercept)  1.604e-01  6.009e-02   2.669  0.00815 ** 
z.lag.1     -3.837e-02  1.462e-02  -2.624  0.00927 ** 
tt           1.283e-04  5.221e-05   2.458  0.01471 *  
z.diff.lag1  4.390e-01  6.498e-02   6.756 1.17e-10 ***
z.diff.lag2  1.017e-01  7.053e-02   1.442  0.15080    
z.diff.lag3 -6.801e-02  7.072e-02  -0.962  0.33719    
z.diff.lag4 -6.709e-02  6.563e-02  -1.022  0.30778    
---
Signif. codes:  0 '***' 0.001 '**' 0.01 '*' 0.05 '.' 0.1 ' ' 1

Residual standard error: 0.01558 on 228 degrees of freedom
Multiple R-squared:  0.2496,    Adjusted R-squared:  0.2299 
F-statistic: 12.64 on 6 and 228 DF,  p-value: 2.659e-12


Value of test-statistic is: -2.6242 4.0776 3.4619 

Critical values for test statistics: 
      1pct  5pct 10pct
tau3 -3.99 -3.43 -3.13
phi2  6.22  4.75  4.07
phi3  8.43  6.49  5.47
\end{verbatim}

\hypertarget{apuxe9ndice-b-dickey-fuller-aumentada---series-primera-diferencia-logaritmos}{%
\subsection{Apéndice B: Dickey-Fuller Aumentada - series primera
diferencia
logaritmos}\label{apuxe9ndice-b-dickey-fuller-aumentada---series-primera-diferencia-logaritmos}}

\begin{Shaded}
\begin{Highlighting}[]
\FunctionTok{map}\NormalTok{(test\_diff, summary)}
\end{Highlighting}
\end{Shaded}

\begin{verbatim}
$ipc

############################################### 
# Augmented Dickey-Fuller Test Unit Root Test # 
############################################### 

Test regression trend 


Call:
lm(formula = z.diff ~ z.lag.1 + 1 + tt + z.diff.lag)

Residuals:
       Min         1Q     Median         3Q        Max 
-0.0311394 -0.0026183 -0.0002759  0.0022865  0.0188296 

Coefficients:
              Estimate Std. Error t value Pr(>|t|)    
(Intercept)  2.493e-03  8.162e-04   3.055  0.00252 ** 
z.lag.1     -6.023e-01  9.967e-02  -6.043 6.13e-09 ***
tt          -1.796e-06  4.998e-06  -0.359  0.71967    
z.diff.lag1  1.560e-02  9.381e-02   0.166  0.86809    
z.diff.lag2  2.671e-02  8.602e-02   0.311  0.75644    
z.diff.lag3  4.315e-02  7.685e-02   0.561  0.57506    
z.diff.lag4 -3.425e-02  6.633e-02  -0.516  0.60618    
---
Signif. codes:  0 '***' 0.001 '**' 0.01 '*' 0.05 '.' 0.1 ' ' 1

Residual standard error: 0.005139 on 227 degrees of freedom
Multiple R-squared:  0.2961,    Adjusted R-squared:  0.2775 
F-statistic: 15.91 on 6 and 227 DF,  p-value: 2.991e-15


Value of test-statistic is: -6.0434 12.1936 18.2894 

Critical values for test statistics: 
      1pct  5pct 10pct
tau3 -3.99 -3.43 -3.13
phi2  6.22  4.75  4.07
phi3  8.43  6.49  5.47


$ipc_subyacente

############################################### 
# Augmented Dickey-Fuller Test Unit Root Test # 
############################################### 

Test regression trend 


Call:
lm(formula = z.diff ~ z.lag.1 + 1 + tt + z.diff.lag)

Residuals:
       Min         1Q     Median         3Q        Max 
-0.0070640 -0.0011132 -0.0002796  0.0007224  0.0096319 

Coefficients:
              Estimate Std. Error t value Pr(>|t|)    
(Intercept)  1.216e-03  4.129e-04   2.945 0.003564 ** 
z.lag.1     -3.288e-01  8.428e-02  -3.902 0.000126 ***
tt          -5.639e-07  2.031e-06  -0.278 0.781523    
z.diff.lag1 -2.828e-01  9.099e-02  -3.108 0.002126 ** 
z.diff.lag2 -3.139e-01  8.425e-02  -3.726 0.000245 ***
z.diff.lag3 -2.721e-01  7.536e-02  -3.611 0.000375 ***
z.diff.lag4 -9.585e-02  6.619e-02  -1.448 0.148997    
---
Signif. codes:  0 '***' 0.001 '**' 0.01 '*' 0.05 '.' 0.1 ' ' 1

Residual standard error: 0.002093 on 227 degrees of freedom
Multiple R-squared:  0.3257,    Adjusted R-squared:  0.3079 
F-statistic: 18.27 on 6 and 227 DF,  p-value: < 2.2e-16


Value of test-statistic is: -3.9017 5.0965 7.6119 

Critical values for test statistics: 
      1pct  5pct 10pct
tau3 -3.99 -3.43 -3.13
phi2  6.22  4.75  4.07
phi3  8.43  6.49  5.47


$ipc_ayb

############################################### 
# Augmented Dickey-Fuller Test Unit Root Test # 
############################################### 

Test regression trend 


Call:
lm(formula = z.diff ~ z.lag.1 + 1 + tt + z.diff.lag)

Residuals:
       Min         1Q     Median         3Q        Max 
-0.0217360 -0.0052347 -0.0004152  0.0052258  0.0290764 

Coefficients:
              Estimate Std. Error t value Pr(>|t|)    
(Intercept)  3.372e-03  1.258e-03   2.680  0.00789 ** 
z.lag.1     -7.176e-01  1.120e-01  -6.406 8.53e-10 ***
tt           1.167e-06  8.152e-06   0.143  0.88630    
z.diff.lag1  1.155e-01  1.024e-01   1.127  0.26073    
z.diff.lag2 -6.746e-02  9.330e-02  -0.723  0.47041    
z.diff.lag3  9.821e-02  7.728e-02   1.271  0.20510    
z.diff.lag4 -5.487e-04  6.624e-02  -0.008  0.99340    
---
Signif. codes:  0 '***' 0.001 '**' 0.01 '*' 0.05 '.' 0.1 ' ' 1

Residual standard error: 0.008422 on 227 degrees of freedom
Multiple R-squared:  0.3603,    Adjusted R-squared:  0.3434 
F-statistic: 21.31 on 6 and 227 DF,  p-value: < 2.2e-16


Value of test-statistic is: -6.4056 13.7103 20.5446 

Critical values for test statistics: 
      1pct  5pct 10pct
tau3 -3.99 -3.43 -3.13
phi2  6.22  4.75  4.07
phi3  8.43  6.49  5.47


$ipc_transporte

############################################### 
# Augmented Dickey-Fuller Test Unit Root Test # 
############################################### 

Test regression trend 


Call:
lm(formula = z.diff ~ z.lag.1 + 1 + tt + z.diff.lag)

Residuals:
      Min        1Q    Median        3Q       Max 
-0.094059 -0.005065  0.000120  0.006471  0.048529 

Coefficients:
              Estimate Std. Error t value Pr(>|t|)    
(Intercept)  3.121e-03  2.165e-03   1.442  0.15080    
z.lag.1     -7.404e-01  9.673e-02  -7.654 5.54e-13 ***
tt          -3.272e-06  1.526e-05  -0.214  0.83040    
z.diff.lag1  1.624e-01  8.887e-02   1.828  0.06894 .  
z.diff.lag2  2.383e-01  8.078e-02   2.950  0.00351 ** 
z.diff.lag3  1.587e-01  7.517e-02   2.111  0.03588 *  
z.diff.lag4  1.081e-01  6.566e-02   1.647  0.10093    
---
Signif. codes:  0 '***' 0.001 '**' 0.01 '*' 0.05 '.' 0.1 ' ' 1

Residual standard error: 0.01575 on 227 degrees of freedom
Multiple R-squared:  0.3026,    Adjusted R-squared:  0.2842 
F-statistic: 16.41 on 6 and 227 DF,  p-value: 1.085e-15


Value of test-statistic is: -7.6541 19.5335 29.2985 

Critical values for test statistics: 
      1pct  5pct 10pct
tau3 -3.99 -3.43 -3.13
phi2  6.22  4.75  4.07
phi3  8.43  6.49  5.47
\end{verbatim}

\hypertarget{apuxe9ndice-c}{%
\subsection{Apéndice C}\label{apuxe9ndice-c}}

\hypertarget{apuxe9ndice-c1-estaduxedsticos-descriptivos-de-las-primeras-diferencias-logaruxedtmicas-del-ipc}{%
\subsubsection{Apéndice C1: Estadísticos descriptivos de las primeras
diferencias logarítmicas del
IPC}\label{apuxe9ndice-c1-estaduxedsticos-descriptivos-de-las-primeras-diferencias-logaruxedtmicas-del-ipc}}

\begin{Shaded}
\begin{Highlighting}[]
\FunctionTok{map\_dfr}\NormalTok{(}
\NormalTok{  ipcs\_log\_diff,}
  \SpecialCharTok{\textasciitilde{}}\NormalTok{tibble}\SpecialCharTok{::}\FunctionTok{tibble}\NormalTok{(}
    \AttributeTok{media    =} \FunctionTok{mean}\NormalTok{(.x, }\AttributeTok{na.rm =} \ConstantTok{TRUE}\NormalTok{),}
    \AttributeTok{varianza =} \FunctionTok{var}\NormalTok{(.x, }\AttributeTok{na.rm =} \ConstantTok{TRUE}\NormalTok{),}
    \AttributeTok{minimo   =} \FunctionTok{min}\NormalTok{(.x, }\AttributeTok{na.rm =} \ConstantTok{TRUE}\NormalTok{),}
    \AttributeTok{maximo   =} \FunctionTok{max}\NormalTok{(.x, }\AttributeTok{na.rm =} \ConstantTok{TRUE}\NormalTok{)),}
  \AttributeTok{.id =} \StringTok{"serie"}\NormalTok{)}
\end{Highlighting}
\end{Shaded}

\begin{verbatim}
# A tibble: 4 x 5
  serie            media   varianza   minimo maximo
  <chr>            <dbl>      <dbl>    <dbl>  <dbl>
1 ipc            0.00377 0.0000308  -0.0333  0.0235
2 ipc_subyacente 0.00339 0.00000575 -0.00112 0.0138
3 ipc_ayb        0.00478 0.0000810  -0.0211  0.0380
4 ipc_transporte 0.00379 0.000313   -0.125   0.0641
\end{verbatim}

\hypertarget{apuxe9ndice-c2-matriz-de-correlaciones-entre-las-primeras-diferencias-logaruxedtmicas-del-ipc}{%
\subsubsection{Apéndice C2: Matriz de correlaciones entre las primeras
diferencias logarítmicas del
IPC}\label{apuxe9ndice-c2-matriz-de-correlaciones-entre-las-primeras-diferencias-logaruxedtmicas-del-ipc}}

\begin{Shaded}
\begin{Highlighting}[]
\NormalTok{df\_ipcs }\OtherTok{\textless{}{-}}\NormalTok{ dplyr}\SpecialCharTok{::}\FunctionTok{bind\_cols}\NormalTok{(ipcs\_log\_diff)}

\NormalTok{cor\_mat }\OtherTok{\textless{}{-}} \FunctionTok{cor}\NormalTok{(df\_ipcs, }\AttributeTok{use =} \StringTok{"pairwise.complete.obs"}\NormalTok{)}
\FunctionTok{round}\NormalTok{(cor\_mat, }\DecValTok{3}\NormalTok{)}
\end{Highlighting}
\end{Shaded}

\begin{verbatim}
                 ipc ipc_subyacente ipc_ayb ipc_transporte
ipc            1.000          0.392   0.600          0.841
ipc_subyacente 0.392          1.000   0.402          0.064
ipc_ayb        0.600          0.402   1.000          0.141
ipc_transporte 0.841          0.064   0.141          1.000
\end{verbatim}

\hypertarget{apuxe9ndice-d-resultados-de-la-estimaciuxf3n-de-modelos-arma11}{%
\subsection{Apéndice D: Resultados de la estimación de modelos
ARMA(1,1)}\label{apuxe9ndice-d-resultados-de-la-estimaciuxf3n-de-modelos-arma11}}

\begin{Shaded}
\begin{Highlighting}[]
\NormalTok{modelos\_arma }\SpecialCharTok{|\textgreater{}} 
  \FunctionTok{glance}\NormalTok{() }\SpecialCharTok{|\textgreater{}} 
  \FunctionTok{select}\NormalTok{(key, .model}\SpecialCharTok{:}\NormalTok{BIC)}
\end{Highlighting}
\end{Shaded}

\begin{verbatim}
# A tibble: 4 x 7
  key            .model     sigma2 log_lik    AIC   AICc    BIC
  <chr>          <chr>       <dbl>   <dbl>  <dbl>  <dbl>  <dbl>
1 ipc            arma11 0.0000257     925. -1842. -1842. -1828.
2 ipc_ayb        arma11 0.0000693     807. -1604. -1604. -1587.
3 ipc_subyacente arma11 0.00000357   1160. -2308. -2308. -2287.
4 ipc_transporte arma11 0.000252      654. -1295. -1295. -1274.
\end{verbatim}

\hypertarget{apuxe9ndice-e-test-ljung-box-autocorrelaciuxf3n-en-residuos}{%
\subsection{Apéndice E: Test Ljung-Box autocorrelación en
residuos}\label{apuxe9ndice-e-test-ljung-box-autocorrelaciuxf3n-en-residuos}}

\begin{Shaded}
\begin{Highlighting}[]
\NormalTok{pruebas\_lb}
\end{Highlighting}
\end{Shaded}

\begin{verbatim}
# A tibble: 4 x 4
  key            .model lb_stat lb_pvalue
  <chr>          <chr>    <dbl>     <dbl>
1 ipc            arma11    11.3     0.501
2 ipc_ayb        arma11    15.7     0.207
3 ipc_subyacente arma11    11.6     0.480
4 ipc_transporte arma11    18.3     0.107
\end{verbatim}



\end{document}
